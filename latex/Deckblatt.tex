\thispagestyle{plain}
\begin{titlepage}

\begin{center}

\LARGE{{\titel}}\\[1.5ex]
%\huge{\textbf{\titel}}\\[1.5ex]
%\LARGE{\textbf{\untertitel}}\\[6ex]
%\LARGE{\textbf{\art}}\\[1.5ex]
%\includegraphics[scale=0.7]{Hft_Logo.png}\\%\\[6ex]
%\includegraphics[scale=0.5]{Swiss_Post_Logo_inverted.png}%\\[6ex]

\end{center}
\footnotesize
\begin{tabular}{w{5cm}p{6cm}}\\\\\\
vorgelegt am:  &  12. Mai 2011\\[1.8ex]

Studienbereich: &  \studienbereich\\
Fakult\"at: &  \mbox{Vermessung, Informatik und Mathematik}\\
Bildungseinrichtung: &  \mbox{Hochschule f\"ur Technik Stuttgart}\\[1.8ex]


von:            &  \autor\\[1.8ex]

Matrikelnummer: &  \matrikelnr\\[1.8ex]

Pr\"ufer:  &  \mbox{\erstgutachter, HfT Stuttgart}\\[1.8ex]


\end{tabular}


\begin{center}
\textbf{Zusammenfassung}

\begin{abstract}
In Zeiten in denen davon gesprochen wird, dass Moore's Law in absehbarer Zukunft
nicht mehr gilt werden zusehends Architekturen eingesetzt, die nicht mehr darauf
vertrauen, dass jede einzelne CPU ausreichend schnell ist, sondern vielmehr 
massiv parallele Rechenleistung bieten. Im Zuge dieses Architekturwandels m�ssen
gleichwertig die eingesetzten Software-Stacks �berdacht und teilweise neu
ausgelegt werden, um die neue Form von Rechenleistung konsolodiert und effizient
zu nutzen.

Clojure ist ein LISP Dialekt und damit eine prim�r funktionale
Programmiersprache, die es sich zum Ziel genommen hat dem Entwickler und der
Anwendung hohe Parallelit�t und Performanz bei gleichzeitig hohem Komfort
bereitzustellen. Diese Arbeit bietet einen �berblick �ber die dahinter stehende
Methodik und grenzt sie gegen den bisherigen Verfahren ab.
\end{abstract}


%\copyright\ \jahr\\[9ex]

\end{center}

\end{titlepage}
