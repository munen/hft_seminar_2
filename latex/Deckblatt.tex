\thispagestyle{plain}
\begin{titlepage}

\begin{center}

\LARGE{{\titel}}\\[1.5ex]
%\huge{\textbf{\titel}}\\[1.5ex]
%\LARGE{\textbf{\untertitel}}\\[6ex]
%\LARGE{\textbf{\art}}\\[1.5ex]
%\includegraphics[scale=0.7]{Hft_Logo.png}\\%\\[6ex]
%\includegraphics[scale=0.5]{Swiss_Post_Logo_inverted.png}%\\[6ex]

\end{center}
\footnotesize
\begin{tabular}{w{5cm}p{6cm}}\\\\\\
vorgelegt am:  &  12. Mai 2011\\[1.8ex]

Studienbereich: &  \studienbereich\\
Fakult\"at: &  \mbox{Vermessung, Informatik und Mathematik}\\
Bildungseinrichtung: &  \mbox{Hochschule f\"ur Technik Stuttgart}\\[1.8ex]


von:            &  \autor\\[1.8ex]

Matrikelnummer: &  \matrikelnr\\[1.8ex]

Pr\"ufer:  &  \mbox{\erstgutachter, HfT Stuttgart}\\[1.8ex]


\end{tabular}


\begin{center}
\textbf{Zusammenfassung}

\begin{abstract}
Clojure ist ein LISP Dialekt, die als gemeinsame Philosophie *Code als Daten*,
*Macros* und *S-Expressions* haben. Clojure ist eine prim�r funktionale
Programmiersprache, die ein reiches Angebot an unver�nderbaren Datenstrukturen
mit sich bringt. Falls ben�tigt, bietet Clojure neue Ans�tze f�r ver�nderbare
Zust�nde - namentlich *Software transactional memory* und *Agenten* f�r
korrekte und �bersichtliche Implementierung von multi-threaded Designs.
\end{abstract}


%\copyright\ \jahr\\[9ex]

\end{center}

\end{titlepage}
