\chapter{Einleitung}
\label{cha:Einleitung}

\section{Motivation}

Foo

\section{Ziel der Arbeit}

Bar

\begin{itemize}

  \item
    So

  \item
    \cite{Python}

  \item
    \NeuerBegriff{Clojure}

\end{itemize}

\Fachbegriff{LISP}

\section{Typographische Konventionen}

Folgende typographische Konventionen sind in dieser Arbeit eingesetzt.

\begin{itemize}
    \item 
        \NeuerBegriff{Neuer Begriff}\\
        Neue Begriffe sind f�r den schnellen �berblick gesondert im Textbild
        hervorgehoben.
    \item 
        \Fachbegriff{Fachbegriff}\\
        Fachbegriffe sind aus dem selben Grund wie neue Begriffe hervorgehoben.
    \item 
        \Eingabe{Eingabe}\\
        Referenzen auf Tastatureingaben sind als solche gekennzeichnet.
    \item 
        \Code{Quellcode}\\
        Quellcode kann wie eine Eingabe im Text eingebettet werden. Bei
        gr��eren Code-Versatzst�cken wird jedoch volles Syntax-Highlighting
        verwendet.
    \item 
        \Datei{C:\bs Pfad\bs Datei}\\
        Pfad- und Datei-Angaben
    \item 
        \Datentyp{Datentyp}\\
        Referenzen auf interne Datenstrukturen und Variablennamen tragen die
        typographische Kennzeichnung Datentyp.
\end{itemize}
