\chapter{Probleme und deren L�sungen in unterschiedlichen Programmiersprachen}
\section{Ruby}
\lstset{language=Ruby, basicstyle=\footnotesize, showstringspaces=false, tabsize=2}
\lstinputlisting[label=lst:ThreadRaceRuby,caption=\texttt{Thread Race Ruby}
]{Listings/thread_race.rb}


Beispielergebnis: 17221

Was ist hier geschehen? Es wurden 10 Threads generiert, die jeweils 10.000x die
Variable 'sum' inkrementieren. Das Problem hierbei ist, dass bei einem solchen
Konzept schnell Dateninkonsistenzen entstehen.

Nehmen wir an Thread \#1 inkrementiert die Variable sum mittels inc(). sum steht
zur Zeit auf dem Wert 100. inc(sum) wird aufgerufen. Nun kommt der Scheduler des
Betriebssystemes und gibt Thread \#2 CPU Zeit. Dieser fuehrt ebenfalls
inc(sum) mit dem Wert 100 aus - sum wird als 101 gespeichert. Der Scheduler
schl�gt wieder zu und gibt Thread \#1 die Chance seine Berechnung zu beenden.
Dieser speichert sum nun ebenfalls als 101

\section{Clojure}

\lstset{language=Lisp, basicstyle=\footnotesize, showstringspaces=false, tabsize=2}
\lstinputlisting[label=lst:ThreadRaceClojure,caption=\texttt{Thread Race Clojure}
]{Listings/tread_race.lisp}
