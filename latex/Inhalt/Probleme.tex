\chapter{Probleme und deren L�sungen in unterschiedlichen Programmiersprachen}
\section{Ruby}
\lstset{language=Ruby, basicstyle=\footnotesize, showstringspaces=false, tabsize=2}
\lstinputlisting[label=lst:ThreadRaceRuby,caption=\texttt{Thread Race Ruby}
]{Listings/thread_race.rb}


Beispielergebnis: 17221

Was ist hier geschehen? Es wurden 10 Threads generiert, die jeweils 10.000x die
Variable \Code{sum} inkrementieren. Das Problem hierbei ist, dass bei einem solchen
Konzept schnell Dateninkonsistenzen entstehen.

Nehmen wir an Thread \#1 inkrementiert die Variable \Code{sum} mittels
\Code{inc()}. \Code{sum} steht zur Zeit auf dem Wert 0. \Code{inc(sum)} wird
aufgerufen. Nun kommt der Scheduler des Betriebssystemes und gibt Thread \#2
CPU Zeit. Dieser fuehrt ebenfalls \Code{inc(sum)} mit dem Wert 0 aus -
\Code{sum} wird als 1 gespeichert. Der Scheduler schl�gt wieder zu und gibt
Thread \#1 die Chance seine Berechnung zu beenden.  Dieser speichert \Code{sum}
nun ebenfalls als 1.

Dieser Effekt wird \NeuerBegriff{Racing Condition} genannt. Er stammt aus dem
Englischen und will aussagen, dass mehrere Entit�ten in einem Wettlauf um
Resourcen stehen. Ohne die Hilfsmittel aus Funktionaler Programmierung m�sste
die Resource \Code{sum} dagegen abgesichert werden. Daf�r werden h�ufig \NeuerBegriff{Locks} eingesetzt. Bei einem Lock wird ein Codebereich gesichert, indem nur einem Thread dazu Zutritt gew�hrt wird. 

Diese Methodologie funktioniert, jedoch muss jede Stelle an der eine Racing
Condition entstehen kann separat gekapselt werden - wird nur eine vergessen
sind Bugs im Programm versteckt, die h�ufig erst sp�t auftauchen. Dar�ber
hinaus ist es mit Locks m�glich sogenannte \NeuerBegriff{Deadlocks}�zu
erschaffen. Ein Deadlock ist eine Situation in der Thread A auf Ressource R
wartet auf die jedoch Thread B Zugriff hat. Thread B jedoch wartet darauf, dass
Thread A die Resourcen R mutiert. Diese Situation kann nicht automatisiert
gel��t werden - das Programm steht.

\section{Clojure}

\lstset{language=Lisp, basicstyle=\footnotesize, showstringspaces=false, tabsize=2}
\lstinputlisting[label=lst:ThreadRaceClojure,caption=\texttt{Thread Race Clojure}
]{Listings/tread_race.lisp}

In diesem Listing ist der Mechanismus des \NeuerBegriff{Software Transactional
Memory} gezeigt. Diesen kennt man aus dem Datenbank-Umfeld. Das Framework
k�mmert sich hierbei darum, dass die Resource \Code{visitors} gesch�tzt wird.
Sie kann nicht ohne einen \NeuerBegriff{dosync} Block mutiert werden. Dieser
Block wiederum stellt wiederum sicher, dass Mutationen transaktional ablaufen. 
