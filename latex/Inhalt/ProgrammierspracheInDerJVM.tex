\chapter{Programmiersprachen in der JVM}
\section{Beispiele}
Neben Java bietet die JVM noch unz�hligen weiteren Programmiersprachen als Laufzeitumgebung. Hier nur die prominentesten Vertreter:

\begin{tabular}{|l|l|l|}
\hline
Groovy		& eine Objektorientierte Skriptsprache. 		& http://groovy.codehaus.org/ 	\\
\hline 
Scala		& eine Sprache mit sowohl Objektorientierten 	& http://www.scala-lang.org/ 	\\	
			& als auch Funktionalen Elementen. 			    & 								\\
\hline 
JRuby		& Implementierung von Ruby 						& http://www.jruby.org/			\\
\hline 
Jython 		& Implementierung von Python 					& http://www.jython.org/		\\
\hline 
Rhino 		& Implementierung von JavaScript 				& http://www.mozilla.org/rhino/	\\
\hline 
AspectJ 	& eine Aspektorientierte Erweiterung 			& http://eclipse.org/aspectj	\\
			& von Java										& 								\\
\hline 
Clojure 	& funktionale Sprache, LISP-Dialekt				& http://clojure.org/			\\
\hline
\end{tabular}
\cite{wiki2}

\section{Vorteile/Nachteile}
Was aber bewegt Rich Hickey und die Entwickler der anderen Sprachen dazu, ihre Sprachen in die JVM zu integrieren oder zu portieren? Es sind die Enormen Vorteile der JVM, die die wenigen Nachteile bei weitem �berwiegen. Dazu z�hlen unter anderem:

\begin{itemize}
  \item \textbf{Plattformunabh�ngigkeit:} Die JVM gibt es f�r sehr viele Plattformen (Linux, Mac, Palm OS, Solaris, Windows, usw.). Das hat zur Folge, dass nur ein Compiler ben�tigt wird: \textbf{Eigene Sprache -> Java Bytecode}.
  \item \textbf{Verbreitung:} Die Verbreitung und Aktualisierung der JVM wird von anderen erledigt. Auch ist der enorme Verbreitungsgrad ein gro�er Vorteil, da die Benutzer so kein neues System installieren m�ssen.
  \item \textbf{Sicherheit:} Genau wie bei Java sind die Programme beim Ablauf vom System gekapselt. Das hei�t sie laufen in einer Art Sandbox und k�nnen so deutlich besser kontrolliert werden.
  \item \textbf{Ressourcenverwaltung:} Da die JVM als Schicht zwischen System und Programm fungiert, sind auch alle Ressourcen gekapselt. Das hei�t man muss sich nur begrenzt um Speicherverwaltung k�mmern. Eine \NeuerBegriff{Garbage Collection} wird auch gleich mitgeliefert.
  \item \textbf{Bibliotheken:} Es kann ohne gro�en Aufwand auf die reichhaltigen Java Bibliotheken zur�ckgegriffen werden, da diese ja meist im \NeuerBegriff{Java Bytecode} vorliegen. So k�nnen existierende APIs weiterverwendet werden, aber auch neue Java Frameworks genutzt werden.
\end {itemize}

Bei all den Vorteilen bringt es aber auch eine paar Nachteile mit sich, eine Sprache in die JVM zu integrieren.

\begin{itemize}
  \item \textbf{Ausf�hrungsgeschwindigkeit:} Durch die zus�tzliche Abstraktionsschicht muss man leider Einbusen bei der Performance hin nehmen. Diese halten sich aber dank zahlreicher Optimierungen in Grenzen.
  \item \textbf{Distanz zur Hardware:} Was auf der einen Seite ein Vorteil ist, kann auf der anderen Seite aber wiederum ein Nachteil sein. Durch die zus�tzliche Abstraktionsschicht ist es leider nicht m�glich, hardwarenah zu programmieren. Das trifft insbesondere auf die Entwicklung von Treibern zu.
\end{itemize}
