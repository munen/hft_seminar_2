% Dokumentenkopf ---------------------------------------------------------------
%   Diese Vorlage basiert auf "scrreprt" aus dem koma-script.
% ------------------------------------------------------------------------------
\documentclass[
    12pt, % Schriftgröße
    DIV10,
    ngerman, % für Umlaute, Silbentrennung etc.
    a4paper, % Papierformat
    oneside, % einseitiges Dokument
    notitlepage, % es wird eine Titelseite verwendet, aber mit Abstract
    parskip=half, % Abstand zwischen Absätzen (halbe Zeile)
    headings=normal, % Größe der Überschriften verkleinern
    listof=totoc, % Verzeichnisse im Inhaltsverzeichnis aufführen
    bibliography=totoc, % Literaturverzeichnis im Inhaltsverzeichnis aufführen
    index=totoc, % Index im Inhaltsverzeichnis aufführen
    captions=tableheading, % Beschriftung von Tabellen unterhalb ausgeben
    final % Status des Dokuments (final/draft)
]{scrreprt}

% Meta-Informationen -----------------------------------------------------------
%   Definition von globalen Parametern, die im gesamten Dokument verwendet
%   werden k�nnen (z.B auf dem Deckblatt etc.).
%
%   ACHTUNG: Wenn die Texte Umlaute oder ein Esszet enthalten, muss der folgende
%            Befehl bereits an dieser Stelle aktiviert werden:
%            \usepackage[latin1]{inputenc}
% ------------------------------------------------------------------------------
\newcommand{\titel}{Clojure}
\newcommand{\untertitel}{und hier kommt der Untertitel}
\newcommand{\art}{Seminararbeit}
\newcommand{\fachgebiet}{Informatik}
\newcommand{\autor}{Benjamin Britsch, Alain M. Lafon}
\newcommand{\studienbereich}{Informatik}
\newcommand{\matrikelnr}{373679, 372991}
\newcommand{\erstgutachter}{Prof. Dr. Stefan Knauth}
\newcommand{\jahr}{2011}
\newcommand{\ort}{Stuttgart, Z�rich}
\newcommand{\logo}{Bilder/Hft_Logo_Klein.png}

\input{Packages}

% Index und Abkürzungsverzeichnisses
\makeindex
\makenomenclature


% Kopf- und Fußzeilen, Seitenränder, etc
\input{Seitenstil}

% eigene Definitionen für Silbentrennung
\include{Silbentrennung}

% eigene LaTeX-Befehle
% Eigene Befehle und typographische Auszeichnungen f�r diese

% Abk�rzungen mit korrektem Leerraum 
\newcommand{\ua}{\mbox{u.\,a.\ }}
\newcommand{\zB}{\mbox{z.\,B.\ }}
\newcommand{\dahe}{\mbox{d.\,h.\ }}
\newcommand{\Vgl}{Vgl.\ }
\newcommand{\bzw}{bzw.\ }
\newcommand{\evtl}{evtl.\ }

\newcommand{\bs}{$\backslash$}

% zum Ausgeben von Autoren
\newcommand{\AutorName}[1]{\textsc{#1}}
\newcommand{\Autor}[1]{\AutorName{\citeauthor{#1}}}

% verschiedene Befehle um W�rter semantisch auszuzeichnen 
\newcommand{\NeuerBegriff}[1]{\textbf{#1}}
\newcommand{\Fachbegriff}[1]{\emph{#1}}

\newcommand{\Eingabe}[1]{\texttt{#1}}
\newcommand{\Code}[1]{\texttt{#1}}
\newcommand{\Datei}[1]{\texttt{#1}}

\newcommand{\Datentyp}[1]{\textsf{#1}}


\begin{document}

% erlaube regulaere Anfuehrungszeichen
\shorthandoff{"}

% subsubsection nummerieren
\setcounter{secnumdepth}{3}
\setcounter{tocdepth}{3}

% Deckblatt und Abstract ohne Seitenzahl
\ofoot{}
\thispagestyle{plain}
\begin{titlepage}

\begin{center}

\LARGE{{\titel}}\\[1.5ex]
%\huge{\textbf{\titel}}\\[1.5ex]
%\LARGE{\textbf{\untertitel}}\\[6ex]
%\LARGE{\textbf{\art}}\\[1.5ex]
%\includegraphics[scale=0.7]{Hft_Logo.png}\\%\\[6ex]
%\includegraphics[scale=0.5]{Swiss_Post_Logo_inverted.png}%\\[6ex]

\end{center}
\footnotesize
\begin{tabular}{w{5cm}p{6cm}}\\\\\\
vorgelegt am:  &  12. Mai 2011\\[1.8ex]

Studienbereich: &  \studienbereich\\
Fakult\"at: &  \mbox{Vermessung, Informatik und Mathematik}\\
Bildungseinrichtung: &  \mbox{Hochschule f\"ur Technik Stuttgart}\\[1.8ex]


von:            &  \autor\\[1.8ex]

Matrikelnummer: &  \matrikelnr\\[1.8ex]

Pr\"ufer:  &  \mbox{\erstgutachter, HfT Stuttgart}\\[1.8ex]


\end{tabular}


\begin{center}
\textbf{Zusammenfassung}

\begin{abstract}
Clojure ist ein LISP Dialekt, die als gemeinsame Philosophie *Code als Daten*,
*Macros* und *S-Expressions* haben. Clojure ist eine prim�r funktionale
Programmiersprache, die ein reiches Angebot an unver�nderbaren Datenstrukturen
mit sich bringt. Falls ben�tigt, bietet Clojure neue Ans�tze f�r ver�nderbare
Zust�nde - namentlich *Software transactional memory* und *Agenten* f�r
korrekte und �bersichtliche Implementierung von multi-threaded Designs.
\end{abstract}


%\copyright\ \jahr\\[9ex]

\end{center}

\end{titlepage}

%\include{Inhalt/Abstract}
\ofoot{\pagemark}

% Vor dem Hauptteil: römische Ziffern
\pagenumbering{Roman}
\tableofcontents % Inhaltsverzeichnis

% Abkürzungsverzeichnis 
\nomenclature{JVM}{Java Virtual Machine}

% für korrekte Überschrift in der Kopfzeile
\clearpage\markboth{\nomname}{\nomname} 
\printnomenclature
\label{sec:Glossar}

\listoffigures % Abbildungsverzeichnis
\addcontentsline{toc}{chapter}{Abbildungsverzeichnis}
\listoftables % Tabellenverzeichnis
\addcontentsline{toc}{chapter}{Tabellenverzeichnis}
\renewcommand{\lstlistlistingname}{Verzeichnis der Listings}
\addcontentsline{toc}{chapter}{Verzeichnis der Listings}
\lstlistoflistings % Listings-Verzeichnis

% arabische Seitenzahlen im Hauptteil 
\clearpage
\pagenumbering{arabic}

\chapter{Einleitung}
\label{cha:Einleitung}

\section{Motivation und Ziel der Arbeit}

Die vorliegende Arbeit ist die schriftliche Ausarbeitung des Seminar2 zum Thema
\NeuerBegriff{Clojure}. In ihr wird Clojure gegen�ber LISP abgegrenzt und �ber
die Vorteile dynamischer und funktionalier Programmierung im Gegenzug zu
konvetiel strukturierter Programmierung besprochen.

\section{Typographische Konventionen}

Folgende typographische Konventionen sind in dieser Arbeit eingesetzt.

\begin{itemize}
    \item 
        \NeuerBegriff{Neuer Begriff}\\
        Neue Begriffe sind f�r den schnellen �berblick gesondert im Textbild
        hervorgehoben.
    \item 
        \Fachbegriff{Fachbegriff}\\
        Fachbegriffe sind aus dem selben Grund wie neue Begriffe hervorgehoben.
    \item 
        \Eingabe{Eingabe}\\
        Referenzen auf Tastatureingaben sind als solche gekennzeichnet.
    \item 
        \Code{Quellcode}\\
        Quellcode kann wie eine Eingabe im Text eingebettet werden. Bei
        gr��eren Code-Versatzst�cken wird jedoch volles Syntax-Highlighting
        verwendet.
    \item 
        \Datei{C:\bs Pfad\bs Datei}\\
        Pfad- und Datei-Angaben
    \item 
        \Datentyp{Datentyp}\\
        Referenzen auf interne Datenstrukturen und Variablennamen tragen die
        typographische Kennzeichnung Datentyp.
\end{itemize}

%\include{Inhalt/Fazit}


\renewcommand{\bibname}{Quellenverzeichnis}
\bibliography{Bibliographie} % Aufruf: bibtex Masterarbeit
\addcontentsline{toc}{chapter}{\bibname}
%\bibliographystyle{alphadin} % DIN-Stil des Literaturverzeichnisses
\bibliographystyle{natdin} % DIN-Stil des Literaturverzeichnisses

% Anhang 
%\begin{appendix}
%    %\clearpage
%    \pagenumbering{roman}
%    \chapter{Anhang}
%    \label{sec:Anhang}
%    % Rand der Aufzählungen in Tabellen anpassen
%    \setdefaultleftmargin{1em}{}{}{}{}{}
%    \input{Anhang}
%\end{appendix}

\newpage

% Index 
%\printindex

\end{document}
